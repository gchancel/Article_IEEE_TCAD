% !TeX spellcheck = en_US
% !TeX root = ./0_article.tex

\section{Introduction}
%\IEEEPARstart{S}{everal} researches have studied Body Biasing Injection (BBI) in the past few years.
%While this injection method had been \textcolor{orange}{paused/forgotten} for a few years, it has recently regained some interest.
%Among the latest studies, a modeling and simulation flow has been proposed, alongside better platforms allowing to achieve greater reproducibility and a deeper analysis of the mechanisms at works in digital integrated circuits subjected to BBI.
%In addition to that

\subsection{State-of-the-art}
	When working with cybersecurity, specifically with hardware security, various fault injection methods are often considered.
	One can point out Electromagnetic Fault Injection (EMFI) \cite{bibid}, Laser Fault Injection (LFI) \cite{bibid}, or Body Biasing Injection (BBI) \cite{bibid}, not to cite them all.
	The current work is dedicated in studying Body Biasing Injection.

	Nowadays, electronic devices are found in every economic sector, and very often they manipulate sensitive data, such as in bank transactions, Internet of Things (IoT) devices, or smartphones.
	To ensure data authenticity, these devices embed cryptographic algorithms.
	While theoretically secure, once implemented on actual devices, these algorithms become fallible, leaking manipulated data, in addition to being sensitive to external disturbances.

\subsection{Fault injection objectives}
	Fault injection methods are set up to perform various malicious manipulation on integrated circuits, such as:
	\begin{itemize}
		\item Denial of service (DoS) \textrightarrow\ Stop circuit operation and the related services;
		\item Verification bypass \textrightarrow\ Modify data on the fly to fake authenticity (e.g. to bypass bootloader security);
		\item Confidential data extraction \textrightarrow\ Modify data to perform differential fault analysis.
	\end{itemize}
