% !TeX spellcheck = en_US
% !TeX root = ./0_article.tex

\section{Introduction}
%\IEEEPARstart{S}{everal} researches have studied Body Biasing Injection (BBI) in the past few years.
%While this injection method had been \textcolor{orange}{paused/forgotten} for a few years, it has recently regained some interest.
%Among the latest studies, a modeling and simulation flow has been proposed, alongside better platforms allowing to achieve greater reproducibility and a deeper analysis of the mechanisms at works in digital integrated circuits subjected to BBI.
%In addition to that

%	\subsection{Context}
	\IEEEPARstart{N}{owadays}, electronic devices are found in every economic sector, and very often manipulate sensitive and confidential data, such as in bank transaction systems, Internet of Things (IoT) devices, smartcards, or smartphones.
	To ensure data authenticity and confidentiality, these devices embed cryptographic algorithms.
	While theoretically secure and robust, once implemented on actual devices, these algorithms become fallible by leaking parts of the manipulated data through various physical quantities such as electromagnetic waves, infrared emissions, or sound emissions, not to cite them all.
	In addition to this, they are sensitive to external disturbances, which, when finely controlled, can lead the circuits to give unwanted information about the manipulated data.

	Cybersecurity, and more specifically hardware security, takes place in this context.
%	In this context, cybersecurity takes place, more specifically hardware security.
	Commonly, when comes hardware security often comes side-channel attacks and fault injection attacks.
	On the one hand, side-channel attacks take advantage of the circuit leakage by measuring the various physical quantities available.
	On the other hand, fault injection aims at inducing controlled physical disturbances into integrated circuits, with methods like Electromagnetic Fault Injection (EMFI) \cite{mathieuEMFIFirst, mathieuEMFI}, Laser Fault Injection (LFI) \cite{lfiFaultModel}, or Body Biasing Injection (BBI) \cite{bbiOrigin}, not to cite them all.
	Among these methods, EMFI and LFI are widely studied and well understood.
	However, despite a resurgence in the past few years, BBI knowledge is still less mature compared to the previously cited methods.
	Therefore, this article is dedicated in presenting our work on Body Biasing Injection, such as the description of better practices to improve BBI reproducibility, in addition to an entire simulation flow for BBI, designed to understand the underlying mechanisms of BBI.

%	\IEEEPARstart{W}{hen} working with cybersecurity, more specifically with hardware security, involving various integrated circuits ranging from smartcards, smartphones, or microcontrollers, various fault injection methods are often considered.
%	We can point out some of the most documented methods such as Electromagnetic Fault Injection (EMFI) \cite{mathieuEMFIFirst, mathieuEMFI}, Laser Fault Injection (LFI) \cite{lfiFaultModel}, or Body Biasing Injection (BBI) \cite{bbiOrigin}, not to cite them all.
%	Our work is dedicated in studying Body Biasing Injection.

	\subsection{Fault injection objectives}
		Before going further in the discussion about BBI, let us first outline the main objectives of fault injection methods.
		Most commonly, fault injection is used to perform various malicious manipulation on integrated circuits, such as:
		\begin{itemize}
			\item Denial of service (DoS) \textrightarrow\ Stop circuit operation and the related services;
			\item Verification bypass \textrightarrow\ Modify data on the fly to fake authenticity (e.g. to bypass bootloader security);
			\item Confidential data extraction \textrightarrow\ Modify data to perform differential fault analysis.
		\end{itemize}
		To achieve these objectives, an attacker can use the previously mentioned injection methods, such as EMFI, LFI or BBI.
		Prior to presenting further our work on BBI, let us analyze the available and existing BBI platforms in the state-of-the-art.
%		\textcolor{red}{To finish.}

	\subsection{BBI in the state-of-the-art}
%		\textcolor{red}{Fait-on un paragraphe sur les plateformes industrielles comme dans la thèse ? Je trouve ça encombrant.}
%
		When compared to EMFI or LFI, BBI has a smaller state-of-the-art, whether in the amount of scientific papers published or in the amount of industrial platforms proposed.
		Currently, there are ten main works focusing on BBI \cite{bbiOrigin, bbiSecond, bbiThird, bbiColin,japbbi, japbbi2, mybbiCosade, mybbiFdtc2022, mybbifdtc2023, colinFdtc2023}.
		Each one of them made a unique contribution for a better understanding of BBI.

		The original works \cite{bbiOrigin, bbiSecond} introduced the technique for the first time by applying its principle to an ASIC embedding counter-measures.
		The researchers performed a Bellcore attack \cite{bellcore}, as this attack has weak fault requirements, on a modular exponentiation partially performed by the target thanks to its arithmetic co-processor.
		
		Then, another paper \cite{bbiThird} introduced an advanced BBI test bench aiming at performing reproducible attacks.
		In addition to this, it introduced a dual-well substrate lumped model to analyze and evaluate electrical phenomena occuring during BBI.
		
%		One year later, another work \cite{bbiSecond} further studied the method, followed by a third work three years later \cite{bbiThird}, introducing an advanced test bench to work and perform attacks with BBI.
		Four years later, another work \cite{bbiColin} demonstrated that BBI can be performed quite easily using Wafer-Level Chip-Scale Packaging (WLCSP), naturally exposing the backside of ICs and thus reducing the complexity to set up BBI attacks.
		In addition to this, it introduced a low-cost tool to perform BBI: the Pico-EMP, a cheap and open-source device able to generate high voltage pulse with simple hardware.
		
%		presented a low-cost BBI platform, dedicated to WLCSP devices \cite{bbiColin}.
		Then, quite interestingly, researchers proposed a study of BBI on flip-chip packaging ICs using an ESD gun as a voltage surge generator, but concluded that its accuracy, either spatially or in terms of analytical capabilities was pretty limited \cite{japbbi, japbbi2}.
		Therefore, the researchers developed a custom tool called a high-voltage pulse injector (HVPI), equipped with a transformer, controller through a NMOS transistor.
		They tested their design against DFF registers and observed bit flips in those.
		
		At the same time, the impact of substrate thickness on BBI efficiency has been studied in \cite{mybbiCosade}, comparing the effects of BBI on identical ICs with their substrate thinned to various levels, in addition to introducing a simulation flow.
		
		Then, \cite{mybbiFdtc2022} studied the impact of dual-well and triple-well substrate types on BBI induced effects, and observed that BBI destructiveness highly depends on the substrate type and the voltage pulse polarity.
		
		Thereafter, \cite{mybbifdtc2023} proposed better practices for BBI platform design while studying a logic gate level model of BBI to further study the injection method impact on ICs.
		
		Eventually, one last work proposed a safety-focused low-cost and open-source design for the practice of BBI \cite{colinFdtc2023}.
%		\textcolor{red}{To FINISH.}

		However, despite this extensive work, there are still unanswered questions, and the current works aims at bringing more answers to the underlying mechanisms of BBI.

%		Before continuing, let us introduce the platform we use for the present work.

%		Before introducing the present work, let us eventually analyze the industrial platforms proposed by various manufacturers and introduce our own test platform.
%		We can distinguish three major actors proposing BBI related products:
%		\begin{itemize}
%			\item Langer EMV-Technik;
%			\item Riscure;
%			\item NewAE Technology.
%		\end{itemize}
%
%		% !TeX spellcheck = en_US
% !TeX root = ./0_article.tex

\begin{figure}
	\centering
	\subfloat[][Langer]{\includegraphics[width=0.2\columnwidth]{./figures/langerBBI.jpg}}
	\hspace{0.1\columnwidth}
	\subfloat[][Riscure]{\includegraphics[width=0.425\columnwidth]{./figures/em-fi-bbi-probe-20-black.jpg}}
	\caption{Langer and Riscure BBI probes.}
	\label{riscure_langer}
\end{figure}
%		\subsubsection{Langer EMV-Technik platform}
%			The German society Langer EMV-Technik proposes an all-in-one and ready-to-use BBI platform composed of two hardware tools:
%			\begin{itemize}
%				\item A current pulse generator with a metal needle, shown in Fig. \ref{riscure_langer}.a;
%				\item A general controller called "Burst Power Station", combining a power supply, control and monitor tool and a software.
%			\end{itemize}
%
%		\subsubsection{Riscure BBI platform}
%			Similar to Ledger, Riscure proposes a complete BBI platform.
%			It is made of two major tools: a generator, originally designed for EMFI probes, and a set of four probes.
%			One of the probes is shown in Fig. \ref{riscure_langer}.b.
%			The generator is a voltage pulse one................................................

%	\subsection{BBI interrogations}
%%		With all the work in the state-of-the-art in mind, there are still remaining questions unanswered about BBI, such as:
%		With all the work in the state of the art in mind, we will try to bring more answers on various points such as:
%		\begin{itemize}
%			\item How to increase the reliability and repeatability of BBI experiments?
%			\item How to model large scale circuits under BBI?
%			\item How BBI induced faults occur?
%		\end{itemize}
%		We will answer these questions through the next section of this work.
