% !TeX spellcheck = en_US
% !TeX root = ./0_article.tex

\section{\textcolor{orange}{Conclusion: to re-write}}
%\section{Conclusion}
	\IEEEPARstart{B}{ody} biasing injection has seen a re-emergence since 2020 \cite{bbiColin}, and various works have brought more and more knowledge throughout the years \cite{bbiOrigin, bbiSecond, bbiThird, bbiColin,japbbi, japbbi2, mybbiCosade, mybbiFdtc2022, mybbifdtc2023, colinFdtc2023}.
	BBI, contrary to EMFI or LFI, uses the silicon substrate of integrated circuits as the main physical medium to interact, transfer energy, and disturb these circuits.
	We introduced, through this work, the cumulative knowledge we gathered concerning BBI.
	This involves various aspects of the subjects.

	First, we studied better platforms aiming at improving BBI experiments repeatability and reliability through low-cost enhancements such as impedance matching and proper grounding.
	We supported these results with actual experiments such as Fault Analysis Mappings and a Differential Fault Attack.

	Then, we introduced large-scale IC modeling thanks to the use of transistor-less models allowing to reduce the computational power required to simulate BBI.

	Eventually, we extended this "transistor-less models" simulation flow to consider the logic functions of integrated circuits under BBI.
	This allowed us to understand the mechanisms at work during fault creation in integrated circuits under BBI, such as data-dependent bit set/reset faults, in addition to understanding the locality of BBI effects.