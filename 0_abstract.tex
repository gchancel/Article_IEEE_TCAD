% !TeX spellcheck = en_US
% !TeX root = ./0_article.tex

\begin{abstract}
	Over the past decades, several fault injection techniques have been investigated, such as Laser Fault Injection (LFI), Electromagnetic Fault Injection (EMFI), or Body Biasing Injection.
%	Each method employs the manipulation of different physical quantities to achieve their objectives.
	Each technique manipulates various physical properties to achieve their objectives.
	EMFI exploits the electromagnetic susceptibility of integrated circuits, LFI uses the photosensitivity of the silicon, and BBI the conductive nature of the silicon substrate in the ICs.
	Although BBI has gained interest over the past few years, there are still numerous aspects to explore.
%	\textcolor{red}{Particularly, methods to set up repeatable BBI platforms and experiment still need to be explored.}nnnnnn
%	In this context, this work aims at presenting our work on BBI in its entirety, such as the set up of better platforms, a differential fault attack using BBI, and an extensive simulation flow dedicated to BBI.
	In this context, this paper presents an enhanced BBI platform, allowing to perform differential fault attacks on a hardware cryptographic co-processor.
	In addition to this, it also describes a simulation flow devloped to provide inshigts on how BBI induces faults in ICs.
\end{abstract}

%\textbf{
%\textcolor{orange}{Orange text is for undecided wording/words.}\\
%\textcolor{red}{Red text is for important messages.}\\
%\textcolor{cyan}{Cyan text is for future bib references to add.}
%}

\begin{IEEEkeywords}
	Article submission, IEEE, IEEEtran, journal, \LaTeX, paper, template, typesetting.
\end{IEEEkeywords}